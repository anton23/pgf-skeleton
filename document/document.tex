\documentclass{article}

\usepackage{graphicx}  
\usepackage{ifthen}
\usepackage{url}

% run pdflatex ``\def\pgf{}\input{report.tex}'' 
% to run pgf commands
\ifthenelse{\not\isundefined{\pgf}}{\usepackage{preamble-tikz}}{}

% the default directory with generated figures is 'generated/'
\ifthenelse{\isundefined{\generated}}{\def\generated{generated}}{}

\makeatletter
\ifthenelse{\not\isundefined{\pgf}}{%
   \newcommand{\inputTikZ}[1]{%
    \beginpgfgraphicnamed{\generated/#1-external}%
    \input{#1.tex}%
    \endpgfgraphicnamed%
    }    
}{%
  \newcommand{\inputTikZ}[1]{%
    \includegraphics{\generated/#1-external.pdf}%
  }  
}
\makeatother

\begin{document}
\title{Minimum working example of my PGF externalisation setup}
\author{Anton Stefanek}
\maketitle
\section{Introduction}
This is a \LaTeX document and Makefile skeleton using externalisation features of \emph{PGF}
\TeX package\footnote{http://sourceforge.net/projects/pgf/} in a slightly non-standard
way. PGF provides various externalisation features that have become increasingly
sophisticated and elegant over the last few years. This skeleton uses them to
allow
\begin{itemize}
    \item Both authors with and without working PGF installation (and
further packages such as \emph{TikZ} and \emph{pgfplots}) to
collaborate on the document. This results in a straightforward export of the document and all generated figure
files that can be used as a submission to journals/conferences (which often
don't allow use of advanced packages such as TikZ and pgfplots).
    \item The use of some \emph{beamer} overlay features in presentations. This
        makes a huge difference in compilation times because each overlay
        requires the whole figure to be typeset again (something that can take
        very long for example when large datasets are included with pgfplots).
\end{itemize}

\section{How it works}

The figures to be externalised are defined in individual files included via the command
\texttt{\\includeTikZ}. This command works in two different ways, based on the option
\texttt{\\pgf}. If defined, the command uses PGF externalisation features
(i.e.\ the \texttt{\\beginpgfgraphicnamed} and related commands) and creates \texttt{-external.pdf} files for
each figure (when ran with the correct \texttt{--jobname} flag). Otherwise, the command just includes the generated pdf files.

The figure pdf files (together with additional pdflatex auxiliary files) are
generated in a separate directory structure (\texttt{.generated/} by default).
When collaborating with an author who doesn't have a working version of PGF,
this directory can be included in version control. Similarly, when submitting to
a journal which does not support some of the drawing packages, it is sufficient
to include the \texttt{generated} directory and omit any of the figure sources
and e.g.\ pgfplots data.

\section{Makefile}

The Makefile watches the figure sources, defined as all the \texttt{.tex} files
in a given list of directories. If a figure changes, pdflatex is ran with the
\texttt{\\pgf} option and an appropriate \texttt{--jobname} flag. 

During the first build, all the external figure pdf files have to be created. In order
to produce each figure pdf file, the whole document has to be built.
Because there are initially no figure files, each of these builds will require
pdflatex to fully draw all the figures that have not been externalised so far.
To avoid this, the Makefile provides the \texttt{make figures} command. This
will initially replace all figures with an empty image and then generates the
figure pdf files one by one.

\section{Examples}
\begin{figure}[hbt]
\centering
\inputTikZ{figures/plot1}
\caption{Sample plot. Source can be found in \texttt{figures/plot1.tex}.}
\label{fig:plot1}
\end{figure}
An equation (referenced from an externalised figure):
\begin{equation}
    y = x^3
    \label{eq:y} 
\end{equation}

\begin{figure}[hbt]
\centering
\inputTikZ{figures/plot2}
\caption{Sample plot, references and citations inside figures work. These might
not get picked up correctly during the first build after the figure is included.
The \texttt{make figure} and \texttt{make figures} commands do this correctly
(and the makefile can be modified to run an additional dummy build before each
figure is externalised). Source in \texttt{figures/plot2.tex}.}
\label{fig:plot2}
\end{figure}

\clearpage
\section{Known issues}

\begin{itemize}    
    \item Each \texttt{.tex} file in \texttt{figures/} has to have a
        corresponding \texttt{inputTikZ} command in the document.
        \texttt{pdflatex} fails if it tries to externalise a figure
        (picked up from a filename in \texttt{figures/}) when there is no
        corresponding \texttt{inputTikZ} command in the document. In case a
        figure is no longer needed, its \texttt{.tex} file has to be moved from
        \texttt{figures/}.
    \item PGF/TikZ overlays 
\end{itemize}

\begin{thebibliography}{9}
    \bibitem{doc}
    Author, \emph{A document}, Publisher.
\end{thebibliography}
\end{document}
